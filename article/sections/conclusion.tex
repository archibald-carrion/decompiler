\documentclass[../main.tex]{subfiles}
\graphicspath{{\subfix{../images/}}}
\begin{document}

Both of our approaches show mixed results. Both figures \ref{fig:minimalist_model_loss}
and \ref{fig:large_model_loss} showcase improvement on the models' prediction 
capability on the training split over time. In particular figure \ref{fig:large_model_loss}
showcases a steep loss decrease curve, avoiding particularly overfitting the validation
split.

Nevertheless, the recall, precision, and $f_1$ measures were very poor on both setups. 
Figure \ref{fig:large_model_metrics} showcases them settling around a score of $2\%$
for the large fine-tuning setup. Even worse, figure \ref{fig:images/training_curves.png}
showcases unstable progress for these metrics in the "minimalist" setup, peaking 
at around $0.6\%$ for precision, $0.4\%$ for recall and $0.45\%$ for the $f_1$ 
score. Granted, these metrics were collected by evaluating the model on one batch
at a time, but one would still expect certain uniformity to appear.

The metrics collected after training are not very promising for both setups. Table
\ref{tab:training_metrics} showcases very poor performance on the "minimalist" setup
model after training. Even worse, table \ref{tab:training_and_evaluation_metrics} 
demonstrates significantly worse metrics on the "large-scale" setup — smaller by
an order of magnitude even, except for perplexity, which is many orders of magnitude
larger and exemplifies more uncertainty for predicting tokens on average.

These observations suggest that fine-tuning the model on a large-scale yielded a
lesser performant LLM for decompilation. Experimental conditions and errors in 
implementation may be to blame. It may also be the case that a considerable amount
of examples were trimmed considerably or possibly entirely due to the maximum input
length for the model.

Nevertheless, there is slight improvement for the "minimalist" setup fine-tuned 
model in comparison to the base model, as suggested by Levenshtein distance and
the BLEU score displayed on table \ref{tab:results}. Further attempts to improving
the experimental setups should consider collecting CodeBLEU \cite{ren2020codebleumethodautomaticevaluation}
and R2I \cite{eom_r2i_2024} scores for judging decompiler quality.

\end{document}